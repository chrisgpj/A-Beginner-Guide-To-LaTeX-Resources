\documentclass[11pt, oneside]{memoir}

\usepackage[]{ragged2e}

\usepackage[]{parskip}
\setlength\parskip{1\baselineskip}

\setSingleSpace{1.05}
\SingleSpacing

\title{A Beginner's Guide to the Documentation of \LaTeX{}}
\author{C.\thinspace G.\thinspace P.\thinspace J.\thinspace}

\usepackage[]{hyperref}

\begin{document}
%\part*{Front Matter, TOC, Introductory Matter}
\maketitle
\thispagestyle{empty}

\RaggedRight

\chapter*{Dedication}
\emph{For Nicole, that we may work together}.

\newpage
\tableofcontents

\chapter{Introduction}
I decided to compile a guide to the documentation of \LaTeX{}. Such a guide is needed, because there are many good works that can help a person to become erudite in \LaTeX{}, but they are not always easy to find; and in every place where a good work is not found, a less satisfactory explanation stands.

This guide is accompanied by (and is a part of) a repository, which is available on github. The name of the repository is ``A Beginner Guide To LaTeX Resources''.

\url{https://github.com/chrisgpj/A-Beginner-Guide-To-LaTeX-Resources}

\subsection{A note on the text}
Throughout this guide I have used ``LaTeX'' in many places, rather than ``\LaTeX{}''. While \LaTeX{} is the official rendering, ``LaTeX'' is official in plain-text environments, and I prefer this in prose.

This booklet is \emph{not} an introduction to LaTeX, and it is \emph{not} to expound the virtues of LaTeX; such things have already been written. This booklet instead attempts to help other people by directing them to documentations and resources that I wish I had on-hand much earlier in my journey.

%\part{One - documentation}
\chapter{\LaTeX{} Documentation}
Without good documentation, there is no such thing as a good system or program, or whatever the \emph{thing} is. It is my view that good documentation cannot save a bad project, but bad documentation can hamstring a good one.

If you read this chapter, you will arrive at the end with the knowledge of where to go to learn about LaTeX, where to lookup commands and markup, and where to go for tutorials on your very first documents.

However, it is necessary to speak briefly on LaTeX first, or else confusion will abound later-on. The following synopsis is the bare minimum.

\section{The Structure of \LaTeX{}}

The place to begin is with \TeX{}.

TeX was made by Donald Knuth to improve the typesetting of academic articles and books, which he noticed to be in decline with the introduction of digital technologies. An axiom of \TeX{} is that is it a \emph{typesetting} program. 

The manual typesetters of old had to forge a path from handwritten manuscripts to formatted print. They had to make decisions about what fonts to use, what types of words to make italic, how to format their chapter-breaks and so on. The manuscript would be 'marked-up' to communicate specific instances of typesetting action; an underlined word usually meant 'make this italic', for example. However, even at this stage there was still a lot of work to be done in figuring out exactly how to execute those instructions — when to break to a new line, when to hyphenate a word, and such. Typesetters were constantly calculating within the overall instructions that they had developed for the text\footnote{I believe that, to this day, we have not surpassed the technical excellence of the typesetters of the 19\textsuperscript{th} century.}

The way TeX works is sometimes likened to the process of typesetting. We write the overall instructions and markup the text using the TeX language. The TeX program then takes those instructions and markup, and transforms these into machine-readable instructions and the engine runs calculations to optimise how the document is rendered, then produces an output, usually in the form of a PDF (portable document format) file.

You will notice people refering to the to different TeX components: the TeX markup/programming language, the TeX program, the TeX typesetting engine. As a whole, TeX might be called a system. These components were all made as part of a synchronous workflow, but there is good reason to consider these elements of the workflow separately, at times.

Outside typographic professionals and programming professionals, most people use \LaTeX{}, not \TeX{}.

LaTeX is a set of macros written for TeX by Leslie Lamport in the early 80's (TeX had its origins in the 70's!). In LaTeX, a few simple lines of code sets you up with a document that would be much more complicated and challenging in TeX. You actually are using TeX in the background, and the typesetting is still based on TeX; you're just using a few LaTeX commands to represent many more TeX commands. If little customisation is introduced to a LaTeX document, it will naturally resemble one of the base templates (that is: the default macros), but it is quite easy to customise the document from this baseline, and in many cases you will even interweave a few commands in base TeX language. Because LaTeX can streamline many parts of the process, we can focus effort where it belongs: on the text itself.

People use LaTeX today because of:
\begin{itemize}
    \item The typography is excellent
    \item The input file is plain text, which comes with a number of advantages. A few are:
    \item[-] You can use markup and commands to indicate formatting, meaning that the content (that is: your writing) is actually separate from the presentation.
    \item[-] Following from the above, the file is often more stable
    \item[-] Being able to 'say what you mean', rather than going through a 'what you see is what you get' system, creates clarity and precision.
    \item It is relatively easy to produce things like tables of contents, bibliographies, and cross-references. LaTeX is considered ideal for thesis-writing because of this.
\end{itemize}

\section*{Official Documentation, or Just Offal?}

Knuth wrote a manual for TeX, just as he has written many books on computer programming. Some people have noted that this is not exactly for a general audience, because it teaches programmers how to typeset, rather than teaching the TeX programming language. However, we need not worry about these things, because we are using LaTeX.

Lamport wrote a manual, ``LaTeX: A Document Preparation System''. Even though it is from 1985, that book is, for the most part, viable. However, very few people seem to use it. It is barely ever referenced on forums, and I suspect that it has simply not been made accessible enough for it to remain at the top of the recommended resource list. The community has stepped in with other resources, some of which are written by people who have made excellent packages for LaTeX and have been deeply involved in its maintenance. I feel that the absence of `official documentation' explains some of the difficulty that people have in learning LaTeX. I hope that this guide saves some time and stress in that regard.

\section{A couple of Favourite Documentations}
So, what has the community produced and made available for free? From what I can see, probably everything you need.

\subsection{Memoir}

One of my favourite contributions is the memoir class documentation. In my own mind, this is the official documentation for LaTeX.

Now, the document class is the \emph{very first thing} you tell LaTeX. It corresponds to the TeX macros that you are working with as your baseline. Memoir is the most popular and well-liked class for writing books and theses. It is the class I use, because it has great specifications for page sizes and margins, and these are things I need to consider when binding. The documentation for memoir is so good that is well-regarded as a general documentation for LaTeX by the broader community. However, some things in it are specific to the class.

The memoir documentation is in the repository, in the  ``Documentations'' folder. 

I like it because it is comprehensive, well-explained, sensitive to broader concepts and theories while still being detailed, does not make too many assumptions about your circumstances, and has excellent cross-referencing to other resources.

I advise you to look at the memoir documentation, but not necessarily to use it as your first starting point. I know that you are a wonderful, intelligent, trailblazing scientist, but one of the below recommendations will still serve you well as the first step; memoir is something to come back to in a bit.

\subsection{A Not So Short Introduction}

The ``Not So Short Introduction to LaTeX'' is a well-established reference guide and introduction to LaTeX. I like it because it acts as a reference book, but then does not skim over things too fast. It genuinely introduces you to LaTeX by carefully thinking about the sorts of things that new LaTeX users would find useful. Consequently, it is indeed not very short — but that is why such things have a table of contents.

I recommend this guide as one of your possible starting points. It will provide you with a good introduction, good reference material, and it will help you to write concise code from the outset.

\subsection{Overleaf}

Continuing our journey to a good beginning-point: Overleaf. Now, Overleaf is not really a documentation source. It is a website that offers a free, online LaTeX editor, and has some pretty good tutorial/reference-guides available as well. I think that you will get on quite well starting with the overleaf editor if you do not have something like VS Code set up; without such a set-up, LaTeX can get a bit overwhelming.

But my purpose here is to examine the documentation side of things. Overleaf has documentation, the first part of which is a tutorial ambitiously named ``learn LaTeX in 30 minutes''. This is a great place to start if you want to get going. As you move forwards, you should transition to also using things like the `Not So Short Introduction', and the memoir class documentation. These documentations are essentially whole books at your disposal. They are more comprehensive than overleaf, and allow some more space to get around the theory in a structured way.

I like overleaf because it is aware of the symbiotic role it plays in the LaTeX community by supplying these beginning points.

\url{https://www.overleaf.com/learn/latex/Learn_LaTeX_in_30_minutes}

\section{Summary So Far}
\begin{itemize}
\item \TeX{} can mean a typesetting language, program, or engine; or all at once (and by reading elsewhere you will discover much more nuance in the components of the \TeX{}).
\item \LaTeX{} is a streamlined way of working with \TeX{} 
\item A well-made online platform, with documentation, like overleaf is a good place for you to start, but you should have more cogent resources on hand as well.
\item The `Not So Short Introduction' is excellent.
\item The memoir class documentation is excellent, and together with the not so short introduction, will provide you with most of the resources needed for understanding \LaTeX{} at a beginner level.
\item \LaTeX{} is an interesting case of a program where the closest thing to the official documentation is not widely used. This has led to a nebulous certainty about where one should go, and that is why I have written this guide. Some documentations provided in this chapter are suitable for beginners, yet are impressively comprehensive, written at a high level by people who have actually contributed to LaTeX.
\end{itemize}

\section{Other Documentations}
\subsection{XeTeX}

\subsection{BiBTeX}

\chapter{Official Sites}

\subsection{The LaTeX Project}

This is the official website of LaTeX in that it is the website of the project responsible for maintaining the entity of LaTeX.

\url{https://www.latex-project.org/}

However, the project funds are actually administered by the Tex Users Group.

\subsection{Tex User Group}

\url{https://www.tug.org/}

This more informal looking website is possibly the most official base for LaTeX and TeX that there is online. 

The Tex Users Group is the administrator of the LaTeX Project.

\subsection{The Comprehensive Tex Archive Network}
\url{https://www.ctan.org}

Example of things you would go to this site for are font catalogues, documentation downloads, and mirrors for packages. It functions as a kind of repository for the TeX Users Group.

\chapter{Online Help}

There are a few places you can go for help online, including tutorial websites and forums. There are only a couple of places I recommend. In terms of tutorials, Overleaf (mentioned above) can be a good resource. As for forums, Stack Exchange is the only one that seems to be staying relevant.

\url{https://tex.stackexchange.com/}

Visit Stack Exchange with caution. Many of the answers are excellent, and many are quite terrible.

I have found the longer documentation for LaTeX to be much more useful than any of the shorter online resources. Online help has been crucial for me, but I have also been through a phase of bad habits, formed by picking out bits and pieces without proper guidance.

I have to admit that my main limitation is not being comfortable in the digital sphere. Too many things seem arbitrary. It is like a world where acronyms abound, without knowledge of their meaning being essential to aptitude. It could be that I prefer referring to written documents, rather than other media, so that I can process the assumptions behind each piece of information. I have not yet found YouTube to be a great resource on LaTeX, but others may get on well there.

\end{document}